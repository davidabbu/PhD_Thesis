\pagebreak
\thispagestyle{empty}
\section*{Resum}

En aquesta tesi doctoral, investiguem la complexa interacció entre les dinàmiques temporals associades amb la memòria i el costum, i els seus efectes en els sistemes socials i econòmics. Per a això, combinem models teòrics, per a explorar les implicacions del costum en models de llindars (pressió social), i l'anàlisi empírica, per a abordar l'impacte dels patrons temporals i espacials en sistemes complexos reals, prenent com a cas d'estudi el mercat immobiliari.

La recerca en aquesta tesi s'estructura en dues parts principals. En la primera part, ens centrem en fer models teòrics per a entendre com l'acostumar-se a un estat (representatiu d'una opinió, comportament, etc.) influeix altres mecanismes de canvi social i quin implicacions té aquest mecanisme en el comportament emergent del sistema. Acostumar-se en aquest context s'entén com una resistència creixent a canviar l'estat actual, la qual cosa també pot entendre's com un record dels estats passats. En altres paraules, com a més temps porti un agent amb un estat, menys probable és que canviï est. En aquesta tesi, analitzem el costum en models de llindars, on el mecanisme de canvi social és la pressió social (modelada com un llindar). Aquests models són usats per a descriure 3 fenòmens socials diferents: segregació, difusió d'innovacions i l'arribada al consens. En el model de segregació de Sakoda-Schelling, els efectes d'acostumar-se s'entenen com una persistència a quedar-se en la residència actual com a més temps un agent hagi estat satisfet allí. Aquesta modificació és capaç de portar el sistema d'un estat mixt a un segregat, per tant, és capaç de trencar la transició de fase mixta-segregada present en el model original. A pesar que el costum promou la segregació, el creixement de dominis en la fase segregada és lent, sent capaç de trencant la invariància temporal. A continuació, introduïm un nou marc matemàtic, estenent l'equació mestra aproximada per a models binaris en xarxes complexes per a incloure els efectes del costum. Aquest marc ens permet escriure en termes d'un conjunt d'equacions diferencials la dinàmica del sistema i entendre que mecanisme rellevant causa el seu estat final. Testem els resultats d'aquestes equacions en el model de Granovetter-Watts, per a investigar com el costum modifica els processos de difusió d'innovacions. Ens trobem que el costum, entesa en aquest model com una resistència a adoptar la innovació, pot alterar significativament la dinàmica de contagi complex del model, on la cascada d'adopció exponencial és reemplaçada per un creixement o exponencial estiratge o en llei de potència, depenent de com modelitzem el costum. Per a aquest model, trobem una solució analítica per a la condició de cascada i els exponents, oferint una comprensió de com el costum i l'estructura de la xarxa influeixen en els processos de contagi complex. Finalment, estudiem un model de llindar simètric, un model on tots dos estats són simètrics i intenten arribar al consens. Els resultats revelen que acostumar-se afecta de manera important en la dinàmica del model, portant a noves fases no presents en la versió original, caracteritzades per un desordre inicial seguit d'un creixement lent de dominis. En aquesta fase, el mecanisme de costum és capaç de portar al consens a l'estat de la minoria inicial. El costum també introdueix un procés de creixement de dominis més lent amb estats transitoris de llarga durada, indicant que els efectes del costum, malgrat promoure l'ordre, poden retardar significativament la convergència del sistema a l'estat estacionari.

En la segona part, passem a una anàlisi empírica de dades reals d'una plataforma en línia de pisos en venda que ens permet analitzar les interaccions espacials i temporals del mercat immobiliari. Usem anuncis que han estat publicats en algun moment durant 2 anys en 3 províncies espanyoles, de manera que puguem oferir una visió objectiva de la dinàmica del mercat, incloent-hi el paper de les agències immobiliàries i la seva influència en els patrons espacials emergents. Comencem explorant la segmentació espacial dins del mercat immobiliari, causada per la presència i influència de les agències immobiliàries. Representem el mercat com una xarxa tripartida que connecta anuncis, agències i cel$\cdot$les espacials, de manera que ens permeti identificar la division del mercat mitjançant diferents algorismes de detecció de comunitats. La nostra anàlisi revela que la segmentació del mercat és consistent a través de diferents resolucions espacials i algorismes, i trobem patrons similars en dades d'Espanya i França (en tots dos països els mercats detectats estan connectats i són més grans que els municipis). A més, quant a la dinàmica temporal, analitzem les ràfegues d'anuncis, els patrons setmanals i la publicació dels anuncis per part de les diferents agències immobiliàries. Observem que la dinàmica dels anuncis exhibeix patrons temporals irregulars, influenciats per un efecte de memòria similar al d'acostumar-se en sistemes socials, però és la probabilitat que un anunci sigui eliminat la que disminueix amb el temps. Aquest efecte de memòria és consistent a través de diferents regions i tipus de propietats, suggerint que és una característica general del mercat immobiliari. També trobem que la publicació d'anuncis per part de les agències està influenciada per la grandària de la seva cartera d'anuncis (preferència per agències grans), el seu preu mitjà (similitud de preus agencia - nou casa) i la seva proximitat espacial (especialització).

\thispagestyle{empty}

En resum, a través de dos punts de vista complementaris, models teòrics i anàlisi empírica, aquesta tesi contribueix a la comprensió de com el costum i la memòria donen forma als sistemes socials i econòmics. Els nostres resultats subratllen el profund impacte de les dinàmiques temporals en els sistemes socioeconòmics, revelant com els efectes no Markovianos alteren els comportaments, portant a nous fenòmens en la dinàmica de la segregació, processos de contagi i problemes de consens. A més, l'anàlisi de dades reals del mercat immobiliari destaca la importància de les dinàmiques temporals (memòria) i espacials (especialització) de les agències immobiliàries en la formació de les estructures del mercat i en els processos de presa de decisions dels venedors. La fortalesa d'aquesta recerca radica en la combinació d'enfocaments teòrics i empírics, basats en l'ús de grans conjunts de dades, teoria de xarxes i models matemàtics simples. Aquest enfocament interdisciplinari crida a futurs desenvolupaments d'aquest tipus, que acaben de començar a revelar els secrets del comportament humà.

\vfill