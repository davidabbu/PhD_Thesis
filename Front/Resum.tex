\pagebreak
\thispagestyle{empty}
\section*{Resum}

En aquesta tesi doctoral, investiguem la complexa interacció entre les dinàmiques temporals associades amb la costum i la memòria i els seus efectes en les dinàmiques socials i econòmiques. Per fer-ho, combinem models teòrics per explorar les implicacions d'acostumar-se en els anomenats models de llindar i l'anàlisi empírica per abordar l'impacte de la memòria i la irregularitat en un sistema complex real: el mercat immobiliari. La investigació es divideix en dues parts principals, cadascuna explorant aspectes crítics de la interacció humana i el comportament econòmic.

A la primera part, ens centrem en models teòrics per aclarir com el costum influeix en altres mecanismes socials i quines són les implicacions per al comportament emergent d'aquests sistemes. Acostumar-se en aquest context s'entén com una resistència creixent a canviar d'estat, que també es pot entendre com una memòria d'estats passats. Analitzem tres models de llindar diferents que simulen diversos fenòmens socials: segregació, difusió de la informació i formació de consens. Comencem amb el model de segregació de Sakoda-Schelling, on incorporem els efectes d'acostumar-se com una creixent persistència en el lloc actual com més temps un agent ha estat satisfet allà. Aquesta modificació condueix un estat mixt cap a la segregació. A més, introduïm un nou marc matemàtic que integra la costum en la dinàmica de models binaris, aplicant-lo al model de Granovetter-Watts per investigar els processos de contagi complex. Trobarem que acostumar-se, entès com una resistència al canvi, pot alterar significativament la dinàmica del contagi complex del model. Finalment, estudiem un model de llindar simètric, un model de consens on ambdós estats són simètrics. Els resultats revelen que acostumar-se impacta dramàticament la dinàmica del model, conduint a noves fases no presents en la versió sense costum, caracteritzades per un desordre inicial seguit d'un coarsening lent.

A la segona part, fem la transició a aplicacions pràctiques utilitzant dades reals de plataformes immobiliàries en línia per analitzar les interaccions temporals en el mercat immobiliari. Utilitzem el conjunt de dades d'Idealista, que cobreix anuncis de les Illes Balears, Barcelona i Madrid, per oferir una visió completa de les dinàmiques del mercat, incloent els rols de les agències immobiliàries i els patrons espacials emergents. Observem que la dinàmica dels anuncis presenta patrons temporals irregulars, influenciats per un efecte de memòria similar al costum. Estenem la nostra anàlisi explorant la segmentació espacial dins del mercat immobiliari, impulsada per la presència i influència de les agències immobiliàries. Mitjançant una representació de xarxa tripartida del mercat, identifiquem submercats robustos a través del clustering per consens de diferents algorismes de detecció de comunitats. La nostra anàlisi revela que la segmentació del mercat és coherent a través de diverses resolucions espacials i algorismes, i trobem patrons similars en conjunts de dades tant d'Espanya com de França.

En general, aquesta tesi proporciona una comprensió integral de com acostumar-se i tenir memoria canvien els sistemes socials i econòmics. Els nostres resultats subratllen l'impacte profund de les dinàmiques temporals en els sistemes socioeconòmics, revelant com els efectes de memòria alteren els comportaments, conduint a noves perspectives sobre les dinàmiques de segregació, processos de contagi i problemes de consens. A més, l'anàlisi de dades del món real del mercat immobiliari destaca la importància de les dinàmiques temporals (memòria) i espacials (especialització) en la configuració de les estructures del mercat i els processos de presa de decisions. Aquest treball contribueix als camps de la ciència social computacional, la teoria de xarxes i la modelització econòmica, proporcionant avenços metodològics i perspectives pràctiques sobre el comportament humà i les dinàmiques del mercat.


\vfill