\pagebreak
\thispagestyle{empty}
\textbf{ \huge Preface}

\vspace*{0.5 cm}

This thesis is an original work that addresses the understanding of temporal interactions in social and economic systems through two distinct approaches: theoretical modeling in social systems and empirical analysis of spatial and temporal dynamics in the housing market. While these two parts are related to each other and approached from a similar perspective, their separation is due to an external factor. Unlike the continuous funding provided by a 4-year grant from the Spanish government, my funding came from various scientific projects. This required me to adapt my work to the research lines of these projects, resulting in a thesis that does not follow a continuous research line with a central problem to address but instead explores different topics and problems.

This diversity, however, is not a drawback. On the contrary, it has allowed me to experience different topics and learn various tools throughout the process. Moreover, in my opinion, addressing both theoretical and practical issues from different points of view enriches its interdisciplinary character.

It is also worth noting that such theses are becoming more common. In my view, the doctorate has evolved from a purely formative period to one focused on developing the skills necessary to work as an academic and researcher. This raises whether it is more useful to compile a manuscript from already reviewed and published scientific articles or to reformulate the system to better suit contemporary needs.

Personally, I have enjoyed writing this thesis, organizing the work in a manner that seems appropriate for a reader and revisiting all the results from these 3 years of work. Nonetheless, writing a thesis requires significant time and effort, and the reward is just a title that allows you to continue doing the same work you did during the previous years - the work of a researcher.



\vfill