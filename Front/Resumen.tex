\pagebreak
\thispagestyle{empty}
\section*{Resumen}

En esta tesis doctoral, investigamos la compleja interacción entre las dinámicas temporales asociadas con la costumbre y la memoria y sus efectos en las dinámicas sociales y económicas. Para ello, combinamos modelado teórico, para explorar las implicaciones de la costumbre en modelos de umbral, y análisis empírico, para abordar el impacto de la memoria y la irregularidad en un sistema complejo real: el mercado inmobiliario. La investigación se estructura en dos partes principales, cada una explorando aspectos críticos de la interacción humana y el comportamiento económico.

En la primera parte, nos centramos en modelos teóricos para aclarar cómo la costumbre influye en otros mecanismos sociales y cuáles son las implicaciones para el comportamiento emergente de estos sistemas. Acostumbrarse en este contexto se entiende como una resistencia creciente a cambiar de estado, que también puede entenderse como una memoria de estados pasados. Analizamos tres modelos de umbral distintos que simulan varios fenómenos sociales: segregación, difusión de la información y formación de consenso. Comenzamos con el modelo de segregación de Sakoda-Schelling, donde incorporamos los efectos de la costumbre como una persistencia creciente en una ubicación actual cuanto más tiempo un agente ha estado satisfecho allí. Esta modificación conduce un estado mixto hacia la segregación. Además, introducimos un nuevo marco matemático que integra la costumbre en la dinámica de estados binarios, aplicándolo al modelo de Granovetter-Watts para investigar los procesos de contagio complejo. Encontramos que la costumbre, entendida como una resistencia al cambio, puede alterar significativamente la dinámica del contagio complejo del modelo. Finalmente, estudiamos un modelo de umbral simétrico, un modelo de consenso donde ambos estados son simétricos. Los resultados revelan que la costumbre impacta dramáticamente la dinámica del modelo, conduciendo a nuevas fases no presentes en la versión sin costumbre, caracterizadas por un desorden inicial seguido de una coalescencia lenta.

En la segunda parte, hacemos la transición a aplicaciones prácticas utilizando datos reales de plataformas inmobiliarias en línea para analizar las interacciones temporales en el mercado inmobiliario. Utilizamos el conjunto de datos de Idealista, que cubre anuncios de las Islas Baleares, Barcelona y Madrid, para ofrecer una visión completa de las dinámicas del mercado, incluyendo los roles de las agencias inmobiliarias y los patrones espaciales emergentes. Observamos que la dinámica de los anuncios presenta patrones temporales irregulares, influenciados por un efecto de memoria similar a la costumbre. Extendemos nuestro análisis explorando la segmentación espacial dentro del mercado inmobiliario, impulsada por la presencia e influencia de las agencias inmobiliarias. Mediante una representación de red tripartita del mercado, identificamos submercados robustos a través del clustering por consenso de diferentes algoritmos de detección de comunidades. Nuestro análisis revela que la segmentación del mercado es coherente a través de diversas resoluciones espaciales y algoritmos, y encontramos patrones similares en conjuntos de datos tanto de España como de Francia.

En general, esta tesis proporciona una comprensión integral de cómo la costumbre y la memoria configuran los sistemas sociales y económicos. Nuestros resultados subrayan el impacto profundo de las dinámicas temporales en los sistemas socioeconómicos, revelando cómo los efectos de memoria alteran los comportamientos, conduciendo a nuevas perspectivas sobre las dinámicas de segregación, procesos de contagio y problemas de consenso. Además, el análisis de datos del mundo real del mercado inmobiliario destaca la importancia de las dinámicas temporales (memoria) y espaciales (especialización) en la configuración de las estructuras del mercado y los procesos de toma de decisiones. Este trabajo contribuye a los campos de la ciencia social computacional, la teoría de redes y la modelización económica, proporcionando avances metodológicos y perspectivas prácticas sobre el comportamiento humano y las dinámicas del mercado.


\vfill