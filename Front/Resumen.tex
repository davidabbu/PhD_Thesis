\pagebreak
\thispagestyle{empty}
\textbf{ \huge Resumen}

\vspace*{0.5cm}

En esta tesis doctoral, investigamos la compleja interacción entre las dinámicas temporales asociadas con la memoria y la costumbre, y sus efectos en los sistemas sociales y económicos. Para ello, combinamos modelos teóricos, para explorar las implicaciones de la costumbre en modelos de umbrales (presión social), y el análisis empírico, para abordar el impacto de los patrones temporales y espaciales en sistemas complejos reales, tomando como caso de estudio el mercado inmobiliario.

La investigación en esta tesis se estructura en dos partes principales. En la primera parte, nos centramos en hacer modelos teóricos para entender como el acostumbrarse a un estado (representativo de una opinión, comportamiento, etc.) influye otros mecanismos de cambio social y qué implicaciones tiene este mecanismo en el comportamiento emergente del sistema. Acostumbrarse en este contexto se entiende como una resistencia creciente a cambiar el estado actual, lo que también puede entenderse como un recuerdo de los estados pasados. En otras palabras, como más tiempo lleve un agente con un estado, menos probable es que cambie este. En esta tesis, analizamos la costumbre en modelos de umbrales, donde el mecanismo de cambio social es la presión social (modelada como un umbral). Estos modelos son usados para describir 3 fenómenos sociales diferentes: segregación, difusión de innovaciones y la llegada al consenso. En el modelo de segregación de Sakoda-Schelling, los efectos de acostumbrarse se entienden como una persistencia a quedarse en la residencia actual como más tiempo un agente haya estado satisfecho allí. Esta modificación es capaz de llevar el sistema de un estado mixto a uno segregado, por lo tanto, es capaz de romper la transición de fase mixta-segregada presente en el modelo original. A pesar de que la costumbre promueve la segregación, el crecimiento de dominios en la fase segregada es lento, siendo capaz de rompiendo la invariancia temporal. A continuación, introducimos un nuevo marco matemático, extendiendo la ecuación maestra aproximada para modelos binarios en redes complejas para incluir los efectos de la costumbre. Este marco nos permite escribir en términos de un conjunto de ecuaciones diferenciales la dinámica del sistema y entender que mecanismo relevante causa su estado final. Testeamos los resultados de estas ecuaciones en el modelo de Granovetter-Watts, para investigar como la costumbre modifica los procesos de difusión de innovaciones. Nos encontramos que la costumbre, entendida en este modelo como una resistencia a adoptar la innovación, puede alterar significativamente la dinámica de contagio complejo del modelo, donde la cascada de adopción exponencial es reemplazada por un crecimiento o exponencial estirado o en ley de potencia, dependiendo de como modelizamos la costumbre. Para este modelo, encontramos una solución analítica para la condición de cascada y los exponentes, ofreciendo una comprensión de como la costumbre y la estructura de la red influyen en los procesos de contagio complejo. Finalmente, estudiamos un modelo de umbral simétrico, un modelo donde ambos estados son simétricos e intentan llegar al consenso. Los resultados revelan que acostumbrarse afecta de forma importante en la dinámica del modelo, llevando a nuevas fases no presentes en la versión original, caracterizadas por un desorden inicial seguido de un crecimiento lento de dominios. En esta fase, el mecanismo de costumbre es capaz de llevar al consenso al estado de la minoría inicial. La costumbre también introduce un proceso de crecimiento de dominios más lento con estados transitorios de larga duración, indicando que los efectos de la costumbre, a pesar de promover el orden, pueden retrasar significativamente la convergencia del sistema al estado estacionario.

En la segunda parte, pasamos a un análisis empírico de datos reales de una plataforma online de pisos en venta que nos permite analizar las interacciones espaciales y temporales del mercado inmobiliario. Usamos anuncios que han estado publicados en algún momento durante 2 años en 3 provincias españolas, de forma que podamos ofrecer una visión objetiva de la dinámica del mercado, incluyendo el papel de las agencias inmobiliarias y su influencia en los patrones espaciales emergentes. Empezamos explorando la segmentación espacial dentro del mercado inmobiliario, causada por la presencia e influencia de las agencias inmobiliarias. Representamos el mercado como una red tripartita que conecta anuncios, agencias y celdas espaciales, de forma que nos permita identificar la division del mercado mediante diferentes algoritmos de detección de comunidades. Nuestro análisis revela que la segmentación del mercado es consistente a través de diferentes resoluciones espaciales y algoritmos, y encontramos patrones similares en datos de España y Francia (en ambos países los mercados detectados están conectados y son más grandes que los municipios). Además, en cuanto a la dinámica temporal, analizamos las ráfagas de anuncios, los patrones semanales y la publicación de los anuncios por parte de las diferentes agencias inmobiliarias. Observamos que la dinámica de los anuncios exhibe patrones temporales irregulares, influenciados por un efecto de memoria similar al de acostumbrarse en sistemas sociales, pero es la probabilidad de que un anuncio sea eliminado la que disminuye con el tiempo. Este efecto de memoria es consistente a través de diferentes regiones y tipos de propiedades, sugiriendo que es una característica general del mercado inmobiliario. También encontramos que la publicación de anuncios por parte de las agencias está influenciada por el tamaño de su cartera de anuncios (preferencia por agencias grandes), su precio medio (similitud de precios agencia - nuevo casa) y su proximidad espacial (especialización).

\thispagestyle{empty}

En resumen, a través de dos puntos de vista complementarios, modelos teóricos y análisis empírico, esta tesis contribuye a la comprensión de como la costumbre y la memoria dan forma a los sistemas sociales y económicos. Nuestros resultados subrayan el profundo impacto de las dinámicas temporales en los sistemas socio-económicos, revelando como los efectos no Markovianos alteran los comportamientos, llevando a nuevos fenómenos en la dinámica de la segregación, procesos de contagio y problemas de consenso. Además, el análisis de datos reales del mercado inmobiliario destaca la importancia de las dinámicas temporales (memoria) y espaciales (especialización) de las agencias inmobiliarias en la formación de las estructuras del mercado y en los procesos de toma de decisiones de los vendedores. La fortaleza de esta investigación radica en la combinación de enfoques teóricos y empíricos, basados en el uso de grandes conjuntos de datos, teoría de redes y modelos matemáticos simples. Este enfoque interdisciplinario llama a futuros desarrollos de este tipo, que acaban de empezar a desvelar los secretos del comportamiento humano.

\vfill