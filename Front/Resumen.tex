\pagebreak
\thispagestyle{empty}
\section*{Resumen}

En los sistemas complejos distribuidos, los sistemas de memoria transaccional distribuida (DTM) son una herramienta muy útil para la programación concurrente. Estos sistemas permiten a los desarrolladores de software escribir código concurrente sin tener que preocuparse por la gestión de la memoria compartida. Además, los DTM ofrecen una interfaz muy sencilla para la programación concurrente, ya que permiten a los desarrolladores de software escribir código concurrente de forma similar a como lo harían si el código fuera secuencial. Sin embargo, los DTM no son una herramienta perfecta, ya que tienen un rendimiento muy inferior al de las estructuras de datos distribuidas. Además, los DTM no son capaces de gestionar estructuras de datos distribuidas de forma eficiente. Por este motivo, los DTM no son una herramienta adecuada para la programación de sistemas distribuidos.

\vfill