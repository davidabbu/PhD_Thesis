\pagebreak
\thispagestyle{empty}
\section*{Abstract}

In this doctoral thesis, we investigate the complex interplay between temporal dynamics associated with aging and memory and their effects on social and economic dynamics. To do so, we combine theoretical modeling, to explore the aging implications in theshold models, and empirical analysis, to adress the impact of memory and burstiness in a real complex system: the housing market. The research is structured into two main parts, each exploring critical facets of human interaction and economic behavior.

In the first part, we focus on theoretical models to elucidate how aging influence other social mechanisms and which are the implications for the emergent behavior of these systems. Aging in this context is understood as an increasing resistance to change state, which can also be understood as a memory to past states. We analyze three distinct threshold models simulating various social phenomena: segregation, information diffusion, and consensus formation. Starting with the Sakoda-Schelling segregation model, we incorporate aging effects as an increasing persistence in a current location the longer an agent has been satisfied there. Instead of disordering, this modification leads a mixed state towards segregation. Furthermore, we introduce a novel mathematical framework integrating aging into binary state dynamics, applying it to the Granovetter-Watts model to investigate complex contagion processes. We find that aging, understood as a resistance to change, can significantly alter the complex contagion dynamics of the model, where the exponential cascade of adoption is replaced by a power-law increase. For the Granovetter-Watts model, an analytical solution was derived for the cascade conditions and exponents, derived from the AME, offering a comprehensive understanding of how aging influences the spread of information or behaviors in social networks. Finally, we study a Symmetrical threshold model, a consensus model where both states are symmetric. The results reveal that aging dramatically impacts the model's dynamics, leading to new phases not present in the non-aging version, characterized by initial disordering followed by slow coarsening. Aging introduces a slower coarsening process and long-lived transient states, indicating that memory effects, despite promoting order, can significantly delay the system's convergence to equilibrium.

In the second part, we transition to practical applications using real data from online real estate platforms to analyze temporal interactions in the housing market. We use the Idealista dataset, covering listings from the Balearic Islands, Barcelona, and Madrid, to offer comprehensive insights into market dynamics, including the roles of real estate agencies and emerging spatial patterns. We observe that the listings' dynamics exhibit irregular temporal patterns, influenced by a memory effect akin to aging, where the probability of a listing being removed decreases over time. This memory effect is consistent across different regions and property types, suggesting it is a general feature of the housing market. We further extend our analysis by exploring spatial segmentation within the housing market, driven by the presence and influence of the real estate agencies. By a tripartite network representation of the market, connecting listings, agencies, and spatial units, we identify robust submarkets via consensus clustering of different community detection algorithms. Our analysis reveals that market segmentation is consistent across various spatial resolutions and algorithms, and we find similar patterns in datasets from both Spain and France (submarkets are connected and larger than municipalities). This segmentation methodology, based on a tripartite network approach, highlights the impact of real estate agencies on shaping spatial submarkets and offers valuable insights for policy-making and economic modeling.

Overall, this thesis via two complementary points of view, theoretical modeling and empirical analysis, provides a comprehensive understanding of how aging and memory shape social and economic systems. Our findings underscore the profound impact of temporal dynamics on socio-economic systems, revealing how memory effects alter behaviors, leading to new insights into segregation dynamics, contagion processes, and consensus problems. Additionally, the analysis of real-world data from the housing market highlights the importance of temporal (memory) and spatial (specialization) dynamics in shaping market structures and decision-making processes. These results advance theoretical understanding and offer practical implications for policy and strategy in social and economic contexts. Our work contributes to the fields of computational social science, network theory, and economic modeling, providing both methodological advancements and practical insights into human behavior and market dynamics.


\vfill