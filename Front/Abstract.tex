\pagebreak
\thispagestyle{empty}
\section*{Abstract}

In this doctoral thesis, we investigate the complex interplay between temporal dynamics associated with aging and memory and their effects on social and economic dynamics. To do so, we combine theoretical modeling, to explore the aging implications in threshold models, and empirical analysis, to address the impact of burstiness and the spatial segmentation in a real complex system: the housing market. 

The research is structured into two main parts. In the first part, we focus on theoretical models to elucidate how aging influence other social mechanisms, and which are the implications of this mechanism to the emergent behavior. Aging in this context is understood as an increasing resistance to change the current state, which can also be understood as a memory to past states. We analyze aging in threshold models, where the social change is driven by peer pressure (modeled as a threshold). These models are used to describe 3 different social phenomena: segregation, information diffusion, and consensus formation. Starting with the Sakoda-Schelling segregation model, we incorporate aging effects as an increasing persistence in a current location the longer an agent has been satisfied there. This modification leads a mixed state towards segregation, even though the coarsening dynamics in the segregated phase are found to be slower. We also introduce a novel mathematical framework, extending the approximate master equation for binary-state dynamics in networks to include aging. This framework allows us to write in terms of a set of differential equations the dynamics of the system and understand the relevant mechanism that drives the system to the final state. We apply test the results of these equations for the Granovetter-Watts model, to investigate how aging modifies complex contagion processes. We find that aging, understood as a resistance to change state, can significantly alter the complex contagion dynamics of the model, where the exponential cascade of adoption is replaced by a stretched exponential or a power-law increase, depending on the aging mechanism. For this model, an analytical solution was derived for the cascade condition and exponents, offering a comprehensive understanding of how aging and the network structure influences the spread of information. Finally, we study a Symmetrical threshold model, a consensus model where both states are symmetric. The results reveal that aging dramatically impacts the model's dynamics, leading to new phases not present in the non-aging version, characterized by initial disordering followed by slow coarsening. In this phase, aging mechanism is able to lead to consensus the state of the initial majority. Aging also introduces a slower coarsening process and long-lived transient states, indicating that aging effects, despite promoting order, can significantly delay the system's convergence to equilibrium.

In the second part, we transition to an empirical analysis using real data from online platforms to analyze the spatial and temporal interactions in the housing market. We use the 2-years dataset, covering listings from 3 Spanish provinces, to offer comprehensive insights into market dynamics, including the roles of real estate agencies and emerging spatial patterns. We start by exploring spatial segmentation within the housing market, driven by the presence and influence of the real estate agencies. By a tripartite network representation of the market, connecting listings, agencies, and spatial units, we identify robust submarkets via consensus clustering of different community detection algorithms. Our analysis reveals that market segmentation is consistent across various spatial resolutions and algorithms, and we find similar patterns in datasets from both Spain and France (submarkets are connected and larger than municipalities). Moreover, regarding the temporal dynamics, we analyze the burstiness of listings', the weakly patterns and the attachment and detachment of listings to the real estate agencies. We observe that the listings' dynamics exhibit irregular temporal patterns, influenced by a memory effect akin to aging, where the probability of a listing being removed decreases over time. This memory effect is consistent across different regions and property types, suggesting it is a general feature of the housing market. We also find that the attachment of listings to agencies is influenced by the agencies' portfolio size (preferential attachment), their mean price (price similarity), and their spatial proximity (specialization).

Overall, this thesis via two complementary points of view, theoretical modeling and empirical analysis, provides a comprehensive understanding of how aging and memory shape social and economic systems. Our findings underscore the profound impact of temporal dynamics on socio-economic systems, revealing how non-Markovian effects associated with aging alter behaviors, leading to new insights into segregation dynamics, contagion processes, and consensus problems. Additionally, the analysis of real-world data from the housing market highlights the importance of temporal (memory) and spatial (specialization) dynamics in shaping market structures and decision-making processes. Our work contributes to the fields of computational social science, network theory, and complex systems modeling, providing both methodological advancements and practical insights to address the behavior human dynamics.


\vfill