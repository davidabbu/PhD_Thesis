\setcounter{page}{1}

This thesis provides a general overview of the research that I have been developing since the beginning of my PhD studies in September, 2021. I could define myself as a curious, creative and open-minded person, following the so called \textit{IFISC attitude}, which means that I am always willing to learn new methods and address new problems, even though they are not directly related to my field of expertise. That is why, through this thesis many topics will be covered, from the study of human behavior and social systems, to the study of complex systems and network theory........

\section{\label{sec:scie_lands} Scientific Landscape}

This thesis adress the study of human behavior and social systems from a \textit{complex systems} perspective, which studies the emergence of collective phenomena that arise from the interactions of many individuals, and that cannot be understood by studying the behavior of individual agents in isolation (the so-called \textit{reductionist} approach) \cite{anderson1972more}. The study of collective phenomena has a long history in the natural sciences, specially in the branch of statistical physics \cite{stanley1971phase}. This branch traditionally studies the emergence of collective phenomena in physical systems, such as the phase transitions in magnetic materials via spin models \cite{onsager-1944}, the turbulence in fluids \cite{frisch1995turbulence}, the synchronization in oscillatory systems \cite{pikovsky2001universal}, or percolation \cite{stauffer-1985}. However, in recent years, the study of complex systems has evolved into the study of emergent phenomena beyond physical systems, such as biological \cite{prigogine1977self}, ecological \cite{may-2001}, economic \cite{arthur-1994}, and social systems \cite{castellano-2009}. From the migration of birds \cite{roche-1997} to the spreading of a fake news through social media \cite{vosoughi-2018}, there are many examples of collective phenomena at which the study of complex systems can be applied.

The cascade of failures in power grids \cite{}, the spread of a disease in a population \cite{anderson1991infectious}, the consensus in political elections \cite{}, the emergence of social norms \cite{} and networks \cite{} are some examples of social collective behavior in which the global phenomena cannot be understood by looking at a single individual. Social and economic collective phenomena has been studied from a variety of perspectives (sociology, psychology, economics, political sciences...), which often relies on qualitative methods, such as interviews, surveys, or ethnographic studies \cite{}. However, the study of social systems from a complex systems perspective aims to provide a quantitative framework to understand the collective behavior, based on methodologies from statistical mechanics and network theory \cite{}. Nevertheless, this approach needs for a large amount of detailed data to validate theories and develop models, which historically has been a limitation for the study of social systems. It is in fact surprising how other branches of physics, where the typical scale of the phenomena is very large, as astrophysics, or very small, as particle physics, do not suffer from a lack of data, while the study of social systems, where the typical scale is human, has been historically limited by the lack of data.

Thankfully, the digital revolution has changed this picture, allowing the storage of large amounts of data generated by human activities, such as social media, mobile phones, or online platforms. Nowadays, every two year, more human socio-economic data is produced than during all the preceding years of human history together. This data, often referred to as \textit{big data}, has opened a new era for the study of social systems at a large scale, together with a paradigm shift in the way we understand human behavior. Nevertheless, this new era comes with an awareness, as the use of big data for the study of human behavior raises important ethical and privacy concerns, which need to be addressed in order to ensure the responsible use of data for the study of social systems. Moreover, from the technical point of view, this huge amount of data needs for a set of computational and mathematical resources to be analyzed and modeled.From this demand, the field of \textit{computational social science} has emerged, with the aim to develop new methods to study human behavior \cite{}. This branch of the complex systems science was born as a combination of methodologies borrowed from social sciences, such as sociology, psychology, or economics, with computational methods from computer science, such as machine learning, data mining, or network theory. This interdisciplinary approach has allowed to develop new methods for forecasting social phenomena and understanding the basic mechanism behind human interactions. 

One can differentiate two main approaches to build a representation of the reality from the data source. The first one is to focus on the prediction and forecasting of a certain social phenomena, such as the spread of a disease or the price of a stock. In this approach, the data is seen as a necessary input to our methodology to make quantitative predictions \cite{}. However, in this approach, the mechanisms behind the phenomena are often hidden in the data, and the model is seen as a black box that provides accurate predictions. In this context, the use of machine learning \cite{} and deep learning \cite{} models are  very popular, as they are able to capture complex patterns in the data. The second approach is to focus on the understanding of the mechanisms behind the phenomena. In this approach, the data is seen as a problem to be understood, an observation from which we can extract qualitative behaviors and patterns. In this context, the aim is to develop very simple models that are able to reproduce the main features of the data, and to extract the basic mechanisms behind the phenomena.

Following the later approach, network science has a critical role in the study of socio-economic systems, as it provides a natural framework to study the interactions between individuals. A network, or graph, is a mathematical representation of a set of nodes (individuals) connected by links (interactions), which allows to study the structure of the interactions and the dynamics of the system. The study of networks has a long history in the natural sciences, from the neurons network in the brain \cite{} to food webs in an ecosystem \cite{}. However, in recent years, new data sources lead to the discovery that complex properties and heterogeneities, present in most social systems, need for a topological descrition in terms of a complex network \cite{}. The definition of a complex network is a network that exhibits non-trivial topological properties, such as a scale-free degree distribution, a small-world property, or a community structure. These properties are often found in social networks, such as the contact network of individuals in a social media platform, the collaboration network of scientists, or the trade network of countries. The study of complex networks has allowed to develop new methods to study the structure of the interactions, the dynamics of the system, and the emergence of collective phenomena. In particular, the study of information spreading as a dynamical system on networks has allowed to understand how information spreads in social networks, and to develop new methods to study information spreading in social networks.


the study of networks has allowed to develop new methods to study the structure of the interactions, the dynamics of the system, and the emergence of collective phenomena. In particular, the study of information spreading as a dynamical system on networks has allowed to understand how information spreads in social networks, and to develop new methods to study information spreading in social networks.


- Also network science has emerged as a new field of study that uses network theory to study human behavior and social systems.

- This perspective is important because it allows us to understand phenomena from a different perspective, and to develop new methods to study human behavior and social systems. 

- For example, the study of information spreading as a dynamical system on networks has allowed us to understand how information spreads in social networks, and to develop new methods to study information spreading in social networks.

- In particular, human interactions exhibit complex activity patterns that are difficult to understand and to model, and that are not present in the study of physical systems.

Early theoretical frameworks for understanding the contagion of ideas were heavily influenced by psychological and sociological theories. Gustave Le Bon's work on crowd psychology in the late 19th century suggested that individuals in a crowd lose their sense of self and, as a result, are more susceptible to the ideas and emotions of the crowd. Later, Gabriel Tarde's laws of imitation proposed that social change is driven by the imitation of behaviors and ideas, a process that is facilitated by close contact and communication between individuals.


\section{\label{sec:Challenges of Computational Social Science} Challenges of Computational Social Science}

- The study of human behavior and social systems is a complex problem that requires the use of computational methods to study human behavior and social systems.

- There are some challenges that are unique to the study of human behavior and social systems, and that are not present in the study of physical systems.

\subsection{\label{subsec:Data availability} Data availability}

- The main problem is the data availability, and the fact that the data that is generated by human activities is not always available for study.

- Notice that the data sources typically used for the study of human behavior does not come from controlled experiments, but from the digital traces that are generated by human activities.

\subsection{\label{subsec:Data analysis} Data analysis}

- The second problem is the data analysis, and the fact that the data that is generated by human activities is not always easy to analyze.

- The data source to analyze usually is a piece of a larger dataset, so we need to be careful to avoid biases in the analysis driven by the data size.

- Temporal windows are also a problem, because when we analyze the dynamics of a system, we need to be careful to avoid biases in the analysis driven by the temporal window.

\subsection{\label{subsec:Modeling} Modeling}

- The third problem is the modeling, and the fact that the data that is generated by human activities is not always easy to model. 

- Deterministic models are not always useful to model human behavior, and we need to use stochastic models to model human behavior.

- Also, mechanistic models and data driven models is something that we need to consider when we model human behavior.

- Another possibility is to use agent-based models to model human behavior.
With the advent of computational methods in the latter half of the 20th century, researchers gained powerful tools to simulate and analyze complex social systems. Agent-based modeling (ABM) emerged as a particularly influential approach, enabling scientists to create and study systems of interacting agents (individuals or collective entities) and observe emergent behaviors from simple rules of interaction. 

\subsection{\label{subsec:Applications} Applications}

- Computational social science has many applications, and it is being used to study human behavior and social systems.

- Sociotechnical systems, social networks, and human dynamics are some of the applications of computational social science.

- fake news detection, information spreading, and social influence are some of the applications of computational social science.

\section{\label{sec:Terminology and general concepts} Terminology and general concepts}

- In this section, we introduce some terminology and general concepts that are used in the study of human behavior and social systems.

Complex networks, interface density, and community structure are some of the concepts that are used in the study of human behavior and social systems.

binary state models, random networks, configuration models, and preferential attachment are some of the models that are used in the study of human behavior and social systems.

\section{\label{sec:Datasets} Datasets}

- We used the idealista dataset

- The strong point of the idealista dataset is that it contains information about the real estate market in Spain, and that it is a large dataset that contains information about the real estate market in Spain.

- The missing point of the idealista dataset is that it contains information about the real estate market in Spain, and that it is a large dataset that contains information about the real estate market in Spain.
