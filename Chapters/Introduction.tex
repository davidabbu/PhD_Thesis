\setcounter{page}{1}

\section{\label{sec:scie_lands} Scientific Landscape}

- How complex systems are studied from a physics perspective, and how the study of complex systems has evolved into the study of social systems.

- Human complex systems ....

- Nevertheless there are some challenges that are unique to the study of social systems, and that are not present in the study of physical systems and the main problem is the data availability.

- After the digital revolution, the amount of data that is generated by human activities has increased exponentially, and this data is being used to study human behavior and social systems. 

- Big data is a term that is used to describe the large amount of data that is generated by human activities, and that is being used to study human behavior and social systems.

- To deal with big data, computational social science has emerged as a new field of study that uses computational methods to study human behavior and social systems.

- Also network science has emerged as a new field of study that uses network theory to study human behavior and social systems.

- This perspective is important because it allows us to understand phenomena from a different perspective, and to develop new methods to study human behavior and social systems. 

- For example, the study of information spreading as a dynamical system on networks has allowed us to understand how information spreads in social networks, and to develop new methods to study information spreading in social networks.

- In particular, human interactions exhibit complex activity patterns that are difficult to understand and to model, and that are not present in the study of physical systems.

Early theoretical frameworks for understanding the contagion of ideas were heavily influenced by psychological and sociological theories. Gustave Le Bon's work on crowd psychology in the late 19th century suggested that individuals in a crowd lose their sense of self and, as a result, are more susceptible to the ideas and emotions of the crowd. Later, Gabriel Tarde's laws of imitation proposed that social change is driven by the imitation of behaviors and ideas, a process that is facilitated by close contact and communication between individuals.


\section{\label{sec:Challenges of Computational Social Science} Challenges of Computational Social Science}

- The study of human behavior and social systems is a complex problem that requires the use of computational methods to study human behavior and social systems.

- There are some challenges that are unique to the study of human behavior and social systems, and that are not present in the study of physical systems.

\subsection{\label{subsec:Data availability} Data availability}

- The main problem is the data availability, and the fact that the data that is generated by human activities is not always available for study.

- Notice that the data sources typically used for the study of human behavior does not come from controlled experiments, but from the digital traces that are generated by human activities.

\subsection{\label{subsec:Data analysis} Data analysis}

- The second problem is the data analysis, and the fact that the data that is generated by human activities is not always easy to analyze.

- The data source to analyze usually is a piece of a larger dataset, so we need to be careful to avoid biases in the analysis driven by the data size.

- Temporal windows are also a problem, because when we analyze the dynamics of a system, we need to be careful to avoid biases in the analysis driven by the temporal window.

\subsection{\label{subsec:Modeling} Modeling}

- The third problem is the modeling, and the fact that the data that is generated by human activities is not always easy to model. 

- Deterministic models are not always useful to model human behavior, and we need to use stochastic models to model human behavior.

- Also, mechanistic models and data driven models is something that we need to consider when we model human behavior.

- Another possibility is to use agent-based models to model human behavior.
With the advent of computational methods in the latter half of the 20th century, researchers gained powerful tools to simulate and analyze complex social systems. Agent-based modeling (ABM) emerged as a particularly influential approach, enabling scientists to create and study systems of interacting agents (individuals or collective entities) and observe emergent behaviors from simple rules of interaction. 

\subsection{\label{subsec:Applications} Applications}

- Computational social science has many applications, and it is being used to study human behavior and social systems.

- Sociotechnical systems, social networks, and human dynamics are some of the applications of computational social science.

- fake news detection, information spreading, and social influence are some of the applications of computational social science.

\section{\label{sec:Terminology and general concepts} Terminology and general concepts}

- In this section, we introduce some terminology and general concepts that are used in the study of human behavior and social systems.

Complex networks, interface density, and community structure are some of the concepts that are used in the study of human behavior and social systems.

binary state models, random networks, configuration models, and preferential attachment are some of the models that are used in the study of human behavior and social systems.

\section{\label{sec:Datasets} Datasets}

- We used the idealista dataset

- The strong point of the idealista dataset is that it contains information about the real estate market in Spain, and that it is a large dataset that contains information about the real estate market in Spain.

- The missing point of the idealista dataset is that it contains information about the real estate market in Spain, and that it is a large dataset that contains information about the real estate market in Spain.
