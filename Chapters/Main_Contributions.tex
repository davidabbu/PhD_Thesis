In conclusion, we offer a comprehensive overview of our most
significant findings and their implications. We have explored in this thesis how aging and temporal dynamics take a relevant role in shaping the behavior in many socio-economic complex systems. We delved into the analysis of aging in threshold models, in particular for a segregation, a complex contagion and a consensus models. This aging implications are found to be crucial in the understanding of the dynamics of these models, leading to new insights and modeling challenges. In the second part of the thesis, keeping this idea of burstiness and memory effects, we analyzed the static and dynamical properties of a real complex system: the housing market. We found that there are irregular temporal patterns in the listings data, which are driven by the strategic interactions between sellers and real estate agencies. Moreover, we identified spatial segmentation in the housing market, showing that the influence of agencies defines robust submarkets. These results provide a novel perspective on the dynamics of the housing market and offer insights into the strategic behavior of agents in this system.

In particular, in Chapter 3, we examined the aging effects in the Sakoda-Schelling segregation model. The study introduced aging as a factor where agents become increasingly resistant to changing their state the longer they remain in it. This modification led to the disappearance of the phase transition between segregated and mixed phases observed in the original model, resulting in segregated states even for high values of tolerance. This indicates that aging promotes segregation by creating a strong attachment to the current location, which counters the tendency to move based on satisfaction alone. The findings suggest significant implications for understanding urban segregation dynamics and policies addressing these.

Chapter 4 focused on deriving the Approximate Master Equation (AME) for binary-state models with aging. The introduction of aging into these models showed that the non-Markovian dynamics significantly affect the system's behavior. Specifically, the AME was used to describe the evolution of the system accurately, highlighting how aging alters the transition rates and the overall dynamics compared to Markovian models. The chapter also discussed potential extensions of this framework to include other non-Markovian effects, such as memory kernels, which could further enrich the understanding of complex systems.

In Chapter 5, we analyzed the impact of aging in the Granovetter-Watts model, a fundamental model for understanding complex contagion processes. Aging was shown to slow down the cascade dynamics without altering the cascade condition. The model's behavior transitioned from exponential to stretched exponential or power-law adoption dynamics, depending on the aging mechanism. This chapter provided analytical expressions for the cascade conditions and exponents, offering a comprehensive understanding of how aging influences the spread of information or behaviors in social networks.

Chapter 6 was divided into two parts, addressing the Symmetrical Threshold model. The first part (Chapter 6A) discussed the ordering dynamics without aging, presenting a detailed analysis of the phase diagram and identifying three distinct phases: mixed, ordered, and frozen. The second part (Chapter 6B) explored the implications of aging in this model. Aging introduced a new dynamical phase for sparse networks characterized by an initial disordering followed by slow ordering. This significantly altered the interface density decay and persistence, showing that aging leads to a slower convergence towards the ordered state and modifies the system's critical behavior.

In Chapter 7, we studied the dynamics of real estate listings in a Spanish dataset, focusing on the decision-making processes of house sellers and the role of real estate agencies. The analysis revealed that the temporal patterns in the listings are driven by various socio-economic factors, including the agencies' influence. The findings highlighted the importance of temporal dynamics in understanding market behaviors and provided insights into the strategic interactions between sellers and agencies.

Chapter 8 extended this analysis by exploring the spatial segmentation of the housing market. By developing a novel methodology to evaluate spatial segmentation using listings data, we identified robust submarkets defined by the influence of real estate agencies. This segmentation was found to be consistent across different spatial resolutions and community detection algorithms, suggesting that market-based spatial divisions are crucial for effective policy-making and economic modeling.

In conclusion, the findings of this thesis underscore the profound impact of aging and temporal dynamics on socio-economic systems. By incorporating aging into various models, we have demonstrated how memory effects alter system behaviors, leading to new insights into segregation dynamics, contagion processes, and market segmentation. These results not only advance theoretical understanding but also provide practical implications for policy and strategy in social and economic contexts. Future research should continue to explore these dynamics, potentially extending the models to incorporate additional non-Markovian effects and applying these insights to other complex systems.
