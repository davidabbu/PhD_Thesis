In this concluding chapter, we offer a comprehensive overview of our most significant findings and their implications. We have explored in this thesis how aging and temporal dynamics take a relevant role in shaping the behavior in socio-economic complex systems. Through this thesis, we have followed two main approaches: the first one is the study of aging in threshold models, and the second one is the analysis of the temporal and spatial dynamics of the housing market from real online listings. We have found that aging and memory effects play a crucial role in the dynamics of these systems, leading to new insights and modeling challenges due to the non-Markovian nature of the dynamics.

\section{Remarks on Aging in threshold models \label{sec:aging_threshold_models}}

In the first part of the thesis, we delved into the analysis of aging in threshold models, in particular 3 different models: a segregation model (Sakoda-Schelling), a complex contagion model (Granovetter-Watts), and a consensus model (Symmetrical Threshold model). The aging is found to be an important modification for both the stationary state and the dynamics of these models. In particular, in Chapter 3, we examined the aging effects in the Sakoda-Schelling segregation model. The study introduced aging as a factor where satisfied agents become increasingly resistant to changing their location the longer they remain satisfied in it. This modification led to the disappearance of the phase transition between segregated and mixed phases observed in the original model, resulting in phase diagram with segregated states even for high values of tolerance. This indicates that aging promotes segregation by creating a strong attachment to the current location, which counters the tendency to move based on satisfaction alone. On the other hand, the coarsening process was found to be slower in the presence of aging, as the interface density decayed with a lower exponent than in the original model. Moreover, we found that the aging mechanism is able to break the time translational invariance of the model, leading to a different scaling of the autocorrelation function. For this model, the full study was conducted via numerical simulations, and there is a need for a more analytical approach to understand the phase diagram of both the original and the version with aging.

Chapter 4 focused on deriving the Approximate Master Equation (AME) for binary-state models with aging. This approach was based on the assumption that the system's dynamics can be described by a set of coupled differential equations, where the age is treated as an additional variable of a node, as the state or the degree. With this framework, we were able to capture the non-Markovian dynamics associated with exogenous and endogenous aging via two mechanisms: aging, when a node remains at its state and increases age, and resetting, when a node remains at its state and resets its age. With this two simple mechanisms, the AME is able to capture the non-Markovian dynamics for a large variety of binary state models with aging. The most important restriction of the AME is that it is assuming an infinite large network with negligible levels of clustering. Nevertheless, this approach could be modified to include network size an stochastic effects related with aging, along the lines of Ref. \cite{peralta-2020B}. In this thesis, Chapter 4 was very useful as a preamble for the analysis of the models presented in the following chapters.

In Chapter 5, we analyzed the impact of aging in the Granovetter-Watts model, a fundamental model for understanding complex contagion processes. Aging was shown to slow down the cascade dynamics without altering the cascade condition. The model's behavior transitioned from exponential to stretched exponential or power-law adoption dynamics, depending on the aging mechanism (exogenous and endogenous). This chapter provided analytical expressions for the cascade conditions and exponents, derived from the AME, offering a comprehensive understanding of how aging influences the spread of information or behaviors in social networks. Furthermore, the results of this model in a Moore lattice were analyzed via numerical simulations, as the AME formalism is not valid for this network structure. In this regular network, the aging mechanism slows down the cascade dynamics dramatically, leading to very slow increasing adoption curves.

Chapter 6 was divided into two parts, addressing the phase diagram and the dynamics of the Symmetrical Threshold model, a consensus threshold model were both states are symmetric. The first part (Chapter 6A) discussed the ordering dynamics without aging, presenting a detailed analysis of the phase diagram and identifying three distinct phases: mixed, ordered, and frozen. The dynamical regimes of the model were characterized by a set of observables that allowed to distinguish the different phases, which were found to be in good agreement with the theoretical predictions. The second part (Chapter 6B) explored the implications of endogenous aging in this model. Aging introduced a new dynamical phase, at the mixed region of the phase diagram, for sparse networks characterized by an initial disordering followed by slow ordering. In this phase, it was found that an initial minority is able to reach consensus due to aging. Moreover, aging significantly altered the interface density decay and persistence, showing that aging leads to a slower convergence towards the ordered state. The AME derived in Chapter 4 was used to analyze the model, providing a comprehensive understanding of the aging effects on consensus dynamics driven of group interactions. For both, the aging and non-aging models, the results were shown for 4 different network topologies (Complete graph, Erd\H{o}s-Rényi, random-regular and Moore lattice), showing how aging implications change for different network structures.

\section{Remarks on Assessing the housing market dynamics \label{sec:housing_market_dynamics}}

In the second part of the thesis, keeping this idea of burstiness and memory effects, we analyzed the static and dynamical properties of a real complex system: the housing market. We found that there are irregular temporal patterns in the listings data, which affect the strategic interactions between sellers and real estate agencies. Moreover, we identified that there is a spatial segmentation in the housing market, driven by the agencies influence and specialization.

In Chapter 7, we studied the dynamics of real estate listings in 3 provinces in Spain, focusing on the listings temporal dynamics, the decision-making processes of house sellers and the role of real estate agencies. The analysis revealed that the temporal patterns in the listings show a memory effect similar to aging, where the probability of a listing being removed decreases with time. This memory effect was found to be consistent across different provinces and property types, suggesting that it is a general feature of the housing market. Additionally, via a temporal bipartite network analysis, we were able to identify the key factors influencing the decision-making process of house sellers when choosing a real estate agency. The results showed that there is a preferential attachment mechanism in the selection of agencies (where the most active agencies tend to attract more listings), a price similarity effect (where listings are posted by agencies with similar prices), and spatial specialization (where agencies tend to focus on specific areas). These findings provide insights about the interactions between sellers and agencies from a new perspective, which might be useful for designing more effective marketing strategies and policies in the housing market.

Motivated by the spatial specialization of the real estate agencies, Chapter 8 extended this analysis by exploring the spatial segmentation of the housing market. By developing a novel methodology to evaluate spatial segmentation using listings data, we identified robust submarkets defined by the influence of real estate agencies. This segmentation was found to be consistent across different spatial resolutions and community detection algorithms. When applied to census level data, we found that the identified communities differ from the ones using cells or municipalitiy level data. To avoid this particular scale-depencency of the detected submarkets, we proposed an stochastically aggregative method to obtain a coherent segmentation from the data at this level. This methodology was proven to be useful and allowed to identify communities in a French dataset, where the identified communities were consistent with the ones obtained from the Spanish dataset. This result suggests that this submarket partition is a result of a universal mechanism and that market-based spatial divisions are crucial for effective policy-making and economic price modeling.

In conclusion, the findings of this thesis underscore the profound impact of aging and temporal dynamics on socio-economic systems. By incorporating aging into various models, we have demonstrated how memory effects alter system behaviors, leading to new insights into segregation dynamics, contagion processes, and consensus problems. Moreover, the analysis of real-world data from the housing market has revealed the importance of temporal (memory) and spatial (specialization) dynamics in shaping market structures and decision-making processes. These results not only advance theoretical understanding but also provide practical implications for policy and strategy in social and economic contexts.
