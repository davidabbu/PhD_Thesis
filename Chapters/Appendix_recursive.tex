%\section{\label{appendix_RR} Internal time recursive relation in Phase ${\rm {\bf I}}$/${\rm {\bf I}}^{*}$}

In Phase I and ${\rm I}^{*}$, the exceeding threshold condition ($m/k > T$) is full-filled for almost all agents in the system. Thus, agents will change their state and reset the internal time once activated. For the original model, all agents are activated once in a time step on average, but for the model with aging, the activation probability plays an important role. We consider here a set of $N$ agents that are activated randomly with an activation probability $p_A(j)$ and, once activated, they reset their internal time. Being $n_i(t)$ the fraction of agents with internal time $i$ at the time step $t$, we build a recursive relation for the previously described dynamics in terms of variables $i$ and $t$:

\begin{eqnarray}
    n_1(t) = \sum_{i=1}^{t-1} p_A(i) \, n_i(t-1) \quad \quad n_i(t) = (1 - p_A(i-1) ) \, n_{i-1}(t-1)  \qquad i > 1. \label{eq:RR1}
\end{eqnarray}

This recursion relation can be solved numerically from the initial condition ($n_1(0) = 1$, $n_i(0) = 0$ for $i > 1$). To obtain the mean internal time at time $t$, we just need to compute the following:

\begin{equation}
    \label{eq:RR}
    \bar{\tau}(t) = \sum_{i=1}^{t} i \, n_i(t).
\end{equation}

The solution from this recursive relation describes the mean internal time dynamics with great agreement with the numerical simulations performed at Phase I (for the complete graph) and Phase ${\rm  I}^{*}$ (for the Erd\H{o}s-R\'enyi and Moore lattice).