\vspace{-1.5cm}
\small
\textbf{The results in this chapter will be published as:}
\vspace{0.05 cm}

% We cite the paper with fontsize 10
\fullcite{abella2024dynamics}
\normalsize
\vspace{0.5 cm}

\section{Introduction}

The advent of digital platforms has revolutionized various industries, and the real estate sector is no exception. The increasing reliance on online listings to buy, sell, and rent properties offers a unique opportunity to analyze the underlying dynamics of the housing market through a computational quantitative approach. Online real estate listings provide a rich source of data that can be used to study the temporal and spatial patterns of property transactions, the behavior of real estate agencies, and the preferences of buyers and sellers. With property portals being nowadays the dominant way to create and access market information, online listings constitute a new type of data to study housing markets \cite{sawyer1999ict, boeing2017new, boulay2021moving}. Scholars studied the spatio-temporal distribution of housing prices \cite{yao2018mapping,adolfsen_segmentation_2022}, revealed the persistence of spatial inequalities in the housing information landscape \cite{boeing2020online}, predicted the social profile of neighborhoods \cite{delmelle2021language}, or detected the segmentation of the market from online search patterns \cite{rae2015online}.
Aside price, pictures or textual descriptions, a listing includes a critical piece of information: the identity of the marketing agency that has posted the listing on the portal. As such, listings constitute digital traces \cite{salganikbit2017} of the work performed by real estate agencies when acquiring, selling or marketing on property portals. It is therefore possible to reconstruct, for each agency, its own portfolio of listings, whose volume and location patterns result from and reflect the heterogeneous practices and market shares of real estate agencies. By informing on \textit{who sells where}, listings offer new ways to examine how real estate agencies unevenly operate and specialize across space \cite{palm1976RealEstate}.

By analyzing these data, we can gain insights about multiple aspects. From an economic standpoint, the specialization of real estate agencies and the correlation of the dynamics with the pricing. From a sociological perspective, we can explore the human activity patterns in online platforms, the temporal dynamics of the listings, and the decision-making process of house owners. From a computational viewpoint, we can investigate the preferential attachment mechanism in the context of real estate agencies, where the likelihood of new listings being posted by an agency depends on the agency's prior activity. However, the dynamics of the housing market, as in any human system, are complex and multifaceted, and understanding them requires a comprehensive analysis of the listings and agencies' behavior.

Previous studies on human dynamics using web analytics have demonstrated that patterns such as preferential attachment and bursty behavior are prevalent in various online activities \cite{szell2010multirelational}. These studies suggest that while individual behaviors might appear unpredictable, aggregate data often reveals systematic trends and patterns. In the context of real estate listings, the temporal patterns of active and total listings on the platform can provide insights into the underlying market dynamics and human activity. The spatial distribution of listings and their correlation with agency activities can help us understand the spatial specialization and operational scope of real estate agencies. By analyzing the listings' dynamics and the agencies' behavior, we can uncover the laws that govern the housing market and provide valuable insights for policymakers, real estate professionals, and researchers. Moreover, the results of this study can be used to develop models of the housing market dynamics, explore how the listings are posted on the online platform, and predict the future evolution of the housing market.

In this chapter, we aim to investigate the laws govern the behavior of real estate listings and agencies. Specifically, we seek to answer the following questions: What are the temporal patterns of active and total listings on the platform? How do these patterns reflect the underlying market dynamics and human activity? Additionally, we examine the spatial distribution of listings and their correlation with agency activities to understand the spatial specialization and operational scope of real estate agencies. We employ statistical analysis to uncover temporal patterns, such as weekly cycles and long-term trends, and to identify the burstiness and heterogeneity of listing activities. Furthermore, we explore the preferential attachment mechanism in the context of real estate agencies, where the likelihood of new listings being posted by an agency depends on the agency's prior activity.

\begin{figure}
    \vspace{0.2 cm}
    \centering
    \includegraphics[width =\textwidth]{Figs/Idealista_dynamics/adds_evo.pdf}
	\caption[Active listings evolution.]{\label{fig:active_adds} Daily number of total listings \textbf{(a)} and active \textbf{(b)} listings during the period 2018 - 2020 for the Barcelona province. The inset in (b) shows a zoomed view to a 1 month period, where the weekly patterns are more evident.}
\end{figure}


\section{Data \label{sec:Data}}

We use the dataset, mentioned in section \ref{sec:Datasets}, of listings published on the portal \texttt{Idealista.com} \cite{idealista}. The dataset covers a 2-year time period, from January 2017 to December 2018 and it comprises a comprehensive collection of online listings geolocated with their (lat, long) coordinates in the Spanish provinces of Balearic Islands, Barcelona, and Madrid. These listings were posted by real estate agencies for renting or selling residential properties. Posted listings by private individuals are not included into the dataset.

Each listing contains a set of attributes such as the price, the surface area, the real estate agency that posted it, its location and the dates of both the publication and the removal of the listing. The dataset also contains information about the type of property (e.g. flat, house, etc.), what allowed us to filter rural parcels and commercial properties. We also removed listings with missing or inconsistent information, such as irregular prices or surface areas. Through this article, the main results are shown for the selling market, but the same analysis stands for the renting market as well.

\section{Listings dynamics}

\begin{figure}
    \centering
    \includegraphics[width =\textwidth]{Figs/Idealista_dynamics/panel_time.pdf}
	\caption[Temporal statistics of the listing dynamics.]{ \label{fig:panel_time} Temporal statistics of the listing dynamics. Listings life (time posted in the platform) distribution for different time windows at Madrid \textbf{(a)}, Barcelona \textbf{(b)} and Balearic Islands \textbf{(c)}. Different colors indicate different time windows. \textbf{(d)} Inter-activations time distribution for the 3 regions. Different colors indicate different region or province. The dashed black line shows a $\nu^{-2}$ power-law distribution. \textbf{(e)} and \textbf{(f)} show the bar sequence of the adds deactivations and activations, respectively, for the Balearic Islands market. The color intensity indicates the number of adds.}
\end{figure}

We begin by analyzing the temporal dynamics of the listings. To do so, we consider as total listings, the listings that have been posted in the online platform before a given ime $t$, and active listings those that are available in the platform at a given time $t$. Fig. \ref{fig:active_adds} shows the daily number of total listings and active listings for the Barcelona province, which shows a representative behavior for all regions. The total number of listings increases linearly with time during the studied period, with no significant variations, showing that new listings are continuously added to the platform. On the other hand, the number of active listings shows a weekly pattern, with a peak on Wednesday and a minimum on the weekends, consistent with human activity patterns in online platforms \cite{szell2010multirelational}. This weekly pattern is more evident when zooming into a 1-month period, as shown in the inset of Fig. \ref{fig:active_adds}(b). Moreover, the number of active listings shows an increasing linear trend, which indicates that the number of listings available in the platform is growing over time. This trend is found in all regions for both renting and selling, what highlights the non-stationary nature of the listings' dynamics.

The active listings dynamics is governed by the listings' life, defined as the number of days a certain listing has been posted in the platform. This measure is a proxy of a listings attractiveness, because when a listing is sold/rented, the agency removes it from the platform. This is an obvious simplification of the reality, since there are many scenarios: a listing removed because a personal decision of the owner, a listing available for long time because the demand is low, etc. However, in this analysis, we assume that when an agency removes a listing, it is because it has been sold/rented, which is the typical scenario. Fig. \ref{fig:panel_time}(a-c) shows the listings' life distribution for the 3 regions studied. The purple line is the distribution for a time window including the whole period (2 years). In all cases, the listings' life distribution shows a fat tail distribution, indicating that, even though there are a lot of listings that are sold in a few days, there are a few present for long periods of time. This heterogeneous life distribution is a signature of the burstiness of the listings' dynamics, what makes this system another example of the irregular human activity patterns mentioned in previous chapters of this thesis.

However, bursty dynamics are usually characterized by fat tail inter-event time distributions. For the listings' dynamics, when we explore the inter-activations time distributions (time between new listings activations) we find a power-law distribution with exponent $\gamma = -2$ for all regions, as shown in Fig. \ref{fig:panel_time}(d). When we compare this distribution with the inter-event time distribution of human activity patterns (with an exponent typically around $\gamma = -1$), we find that the listings' dynamics are less heterogeneous. A similar behavior is found in Fig. \ref{fig:panel_time}(e-f), where we show the bar sequence of the adds deactivations and activations for the Balearic Islands market. Even though we differentiate an irregular temporal dynamics on top of the weakly pattern, with days when there are more listings added or removed, the overall activation behavior is not as bursty as in other human activities. Therefore, the listings' dynamics show a high heterogeneous life distribution, but a less heterogeneous inter-event time distribution.

\begin{figure}
    \vspace{0.2 cm}
    \centering
    \includegraphics[width = 0.8\textwidth]{Figs/Idealista_dynamics/temporal_bipartite.pdf}
	\caption[Housing market as a temporal bipartite network.]{Schematic representation of the temporal bipartite network between listings and agencies. On top, there is a temporal period, represented as a black box, in which the different listings are represented by colored horizontal lines, a representation of the time listings have been active in the platform. The color of the listings represent the real estate agency that posted them. For each time window, represented by purple and grey boxes (from $t_0$ to $t_f$), we build a bipartite network where listings are connected to the agency that posted it if the listing is active inside the temporal window. The colored big circles represent the agencies' nodes and the black small circles, the listings active posted. \label{fig:temporal_bipartite}}
\end{figure}

This process can be understood with the idea of ``aging'': Listings are added to the platform at some constant rate, highlighted by the linear increase of the total listings. But, the longer a listing is in the platform, the less likely it is to be sold/rented, motivated by the fat tail life distribution. In this particular scenario, aging does not reflect the attachment to a previous belief, as in the models in the previous part. Instead, aging acts as a time-changing attractiveness of the listing. This stochastic process is known in the literature as \textit{delayed degradation}, in which particles are added to a system at a constant rate, and they die after a certain time $\tau$ after being created. This time $\tau$ is usually allowed to be randomly distributed. This process has been studied in many situations, such as gene regulation \cite{}, neuronal activity \cite{}, and physiological processes \cite{}. In general, the delayed degradation stochastic process is solved in Ref. \cite{LaFuerza2013} but only for life distributions with a well-defined mean. In our case, the life distribution is fat-tailed, and the delayed degradation process needs to be treated with more detail.

\section{Agencies dynamics}

In previous section, we have analyzed the listings' dynamics, just exploring how the total and active adds increase in size. However, the listings are not isolated entities, but they have a price, a location and are posted by a real estate agency. In this section, we focus on the decision-making process that house owners follow when they decide to sell/rent their properties through a real estate agency, and how it is correlated with the pricing and the location of the listings.

To this end, we build a temporal bipartite network between listings and agencies. At each time window, defined by an initial date $t_0$ and a final date $t_f$, we construct a bipartite network where listings are connected to the agency that posted it if the listing is active in any time $t'$, inside the temporal window $t_0 \leq t' \leq t_f$. Does not matter if the listing was removed after $t_f$ or if it was added before $t_0$, we consider active listings in the time window. Note that this network is very simple, as a listing cannot be connected to more than one agency. As it is observed in the schematic representation of the temporal bipartite network in Fig. \ref{fig:temporal_bipartite}, as the window moves in time, the bipartite network changes structure, allowing us to infer the dynamics listings-agencies.

\subsection{Preferential attachment}

\begin{figure}
    \centering
    \includegraphics[width =\textwidth]{Figs/Idealista_dynamics/panel_degree.pdf}
	\caption[Preferential attachment to agencies.]{\label{fig:panel_degree}Preferential attachment to agencies. \textbf{(a)} Degree distribution of the agencies in the 3 regions. \textbf{(b)} Average degree increase of an agency as a function of the degree (listings posted) of the agency previously to the attachment. For both plots, different colors and markers indicate different regions or provinces. In \textbf{(a)}, the black dashed line shows a power-law distribution with exponent $\alpha  =-2$, and in \textbf{(b)} shows a linear increase with fitted slope $C = 0.04$.}
\end{figure}

Considering the cumulated network of all the listings posted during the whole period, we can analyze the degree distribution of the agencies. The degree of an agency in this context is the number of listings ever posted by the agency. As shown in Fig. \ref{fig:panel_degree}(a), the degree distribution of the agencies in the 3 regions follows a power-law distribution. This result suggests that agencies follow a preferential attachment mechanism, where the probability of a new listing being attached to an agency is proportional to the number of listings posted by the agency previously. This is a common mechanism in many real-world networks, such as the World Wide Web \cite{barabasi1999emergence}, citation networks \cite{redner1998popular}, and social networks \cite{barabasi1999emergence}. In the case of the housing market, this mechanism is a signature of the reputation of the agencies, as the more listings an agency has, the more likely it is to have new listings attached to it.

To verify if the underlying mechanism is indeed preferential attachment, we analyze the average degree increase of an agency as a function of the degree of the agency previously to the attachment. For this purpose, we build a bipartite network for a time window of 1 month and compute the increase of degree of the agencies after a week. We repeat this process for all the time windows in the period, and we average the results. As shown in Fig. \ref{fig:panel_degree}(b), the average degree increase of an agency is linearly correlated with the degree of the agency previously to the attachment. This result confirms that the agencies follow a preferential attachment mechanism, where the probability of a new listing being attached to an agency is proportional to the number of listings posted by the agency previously. However, the popularity of the agency is not the only key factor in the attachment process. The price of the listings and the location of the listings are also important factors that influence the decision-making process of the house owners.

\begin{figure}
    \centering
    \includegraphics[width =\textwidth]{Figs/Idealista_dynamics/labeled_sigma_price.pdf}
	\caption[Variance of the agency price vs mean agency price.]{Variance of the listings price posted by an agency as a function of its mean price for the price per square meter \textbf{(a)} and the total price \textbf{(b)}. Different colors and markers indicate different regions or provinces. The black dashed line shows a linear fit with $\sigma = p^{{m}^2}$ for both plots. \label{fig:sigma_price}}
\end{figure}

\subsection{Price correlations}

Since we have both the price and the surface area (in square meters), we can compute the price per square meter of the listings, which is a common metric used in the real estate market \cite{}. Both the price per square meter and the total price of the listings are distributed according to a log-normal distribution, an expected result in economic systems like ours \cite{}. The price range of an agency can be defined by the mean price of the listings and its variance. Fig. \ref{fig:sigma_price} shows the correlation between the variance and the mean of the agency prices. For both, the price per square meter and the total price, the variance of the listings price posted by an agency is proportional to the mean price of the agency, such that the higher the listings prices of an agency, the higher the variance. This proportionality has been observed for population count in ecological contexts \cite{Violeta cites}, and also recently in social systems \cite{Violeta citation}, and it is known as Taylor's law \cite{}. In physics literature, this proportionality is known as fluctuation scaling \cite{}, and it is shown to be related with heavy tailed data \cite{PNAS_Brown_2021}, highlighting the heterogeneity in agencies pricing.

\begin{figure}
    \centering
    \includegraphics[width =\textwidth]{Figs/Idealista_dynamics/panel_price.pdf}
	\caption[Price segmentation by the degree.]{\textbf{(a)} Average degree (number of listings) of an agency as a function of its mean price per square meter. Different colors and markers indicate different regions or provinces. Degree distribution among the agencies \textbf{(b)-(c)-(d)} and price histograms for 60 representative agencies \textbf{(e)-(f)-(g)} at the different price segments: \textbf{(b)-(e)} $p^{{m}^2} < 800 \, \textup{\euro} / m^2$, \textbf{(c)-(f)} $800 \, \textup{\euro}  / m^2 < p^{{m}^2} < 10^4 \, \textup{\euro}  / m^2$, and \textbf{(d)-(g)} $p^{{m}^2} > 10^4 \, \textup{\euro}  / m^2$. These plots correspond to the Madrid housing market. For the price distributions, the colors indicate histograms for different agencies. \label{fig:panel_price}}
\end{figure}

Besides price itself, we analyze how correlated is the price of the listings of an agency with its degree in the bipartite network for all the period. As shown in Fig. \ref{fig:panel_price}(a), the average degree of an agency shows a non trivial dependence with the price per square meter. For low mean price, the degree values are very low. If we increase the price, we reach the typical prices in the market and degree increases very fast to a certain value that is maintained during a range of prices. For very high prices, the degree decreases to similar values as for low prices. This result suggests that agencies with low prices have few listings, and agencies with very high prices have also few listings. The agencies with the highest number of listings are those with prices in the typical range of prices in the market. This degree dependence allows us to segment the system in different price ranges according to how large is the mean degree. At the low prices segment (Fig. \ref{fig:panel_price}(b)-(e)), the degree distribution is homogeneous, and the agencies show a price distribution peaked at a certain price and with low fluctuations. For middle prices, arround $10^3 - 10^4$ \euro / m$^2$ (Fig. \ref{fig:panel_price}(c)-(f)), the degree distribution exhibits the power-law behavior observed in Fig. \ref{fig:panel_degree}(a). In this price segment, the agencies show price distributions similar to a log-normal distribution, with a peak at a certain price and higher fluctuations. For high prices (Fig. \ref{fig:panel_price}(d)-(g)), the degree distribution is rather homogeneous and the price distribution of the agencies shows higher variance, since the agencies include both high prices and middle prices listings. Note that this results are consistent with the fluctuation scaling observed in Fig. \ref{fig:sigma_price}. This segmentation is an effect of the coexistence of generalist agencies, that post agencies in a wide range of prices, and specialist agencies, which focus on a specific submarket segment.

So far, the price correlations described are analyzed from a static point of view. Regarding price dynamics, we explore how the preferential attachment is correlated with the price per square meter of the listings. As we did for the degree correlation, we compute the mean agency price during a time window of 1 month and price of the new listings attached to the agency during the next week. This process is repeated for all the time windows in the period. When we average over possible values of the new listings price, we observe a positive correlation, similar to linear, between the mean agency price and the new listings attached price (see Fig. \ref{fig:attach_price}(a)). This correlation seems to be robust across the different regions, highlighting how listings with a certain price attach to agencies that work in a similar price range.

\begin{figure}
    \centering
    \includegraphics[width =\textwidth]{Figs/Idealista_dynamics/panel_attach_price.pdf}
	\caption[Attachment dynamics of new listings by price.]{Attachment of new listings by price. \textbf{(a)} Mean agency price per square meter as a function of the price of the new listing attached to the agency. Different colors and markers indicate different regions or provinces. The black dashed line shows a linear increase $\langle p^{{m}^2} \rangle = p^{{m}^2}$. \textbf{(b)} Distribution of the logarithmic difference between the price of the new listing and the mean price of the agency. Different colors indicate different regions or provinces. Each solid colored line corresponds to a T-student fit for each region. The parameters are: $\mu = 0.02$, $\sigma = 0.5$ for Madrid, $\mu = 0.02$, $\sigma = 0.5$ for Barcelona, and $\mu = 0.02$, $\sigma = 0.5$ for the Balearic Islands. \label{fig:attach_price}}
\end{figure}

For each new attachment, we can compute how the price of a new listing attached fluctuates around the mean price of the agency. Fig. \ref{fig:attach_price}(b) shows the distribution of the logarithmic difference between the price of the new listing and the mean price of the agency. This distribution is found to be centered at $0$, as expected from the previous result, and is well-fitted by a T-student distribution. Therefore, besides the preferential attachment, the listing price proportionality is an important factor of the housing market dynamics. When a new house is selled, is more likely to be sold by an agency with both, a high number of listings (degree) and a similar price range, two mechanisms that are correlated and will compete in the decision-making process.

\subsection{Specialization in space}

It is said that in the real estate market, the three most important factors are location, location, and location \cite{}. In this section, we follow a similar approach to previous section, but now focussing on the spatial correlations of the listings. The location of the listings is well-defined by its coordinates, but the location of an agency is not so clear. We can define as the agency location in a time window, the center of mass (CM) of the active listings posted by the agency in that time window. This definition is a simplification, as the location of an agency is not well-defined, but it allows us to explore the spatial correlations of the listings via the distance from the listings and the agency center of mass. Fig. \ref{fig:distance_panel}(a) shows the distribution of the distance between listings and the agency center of mass for the 3 regions. This distribution is computed for all the listings in the period, is peaked around $10^2$ m, and follows an exponential decay. In fact, for Barcelona and Madrid, the largest distance is around $10^5$ m, but for the Balearic Islands, we observe a longer tail that reaches $2.5 \times 10^5$ m. This occurs because of the natural spatial segmentation of the Balearic Islands and the presence of generalist agencies that post listings in different islands.

\begin{figure}
    \centering
    \includegraphics[width =\textwidth]{Figs/Idealista_dynamics/distance_panel.pdf}
	\caption[Distance correlations.]{\textbf{(a)} Distribution of the distance between listings of a certain agency and the agency center of mass (mean location of the listings posted by the agency). Different colors indicate different regions or provinces. \textbf{(b)} Variance of the listings distance to the agency center of mass of an agency as a function of the mean distance of that same agency. Different colors indicate different regions or provinces. The black dashed line shows a linear increase $\sigma(d) = \langle d \rangle$. \label{fig:distance_panel}}
\end{figure}

As it occurred for the price of listings, the variance of the distance between the agency CM and the listings of an agency is proportional to the mean distance of the agency. This proportionality is shown in Fig. \ref{fig:distance_panel}(b) for the 3 regions studied, even though the relation is not as linear as for the price, specially for low mean agency distances. This result suggests that agencies with a higher radius of action $R$, defined as the mean distance $R = \langle d \rangle$, have a higher heterogeneity in the listings' location. Moreover, this process highlights the absence of a typical size of an agency in terms of a radius, going from generalist agencies that operate effectively everywhere, to specialist agencies that focus on a specific region.

\begin{figure}
    \centering
    \includegraphics[width =\textwidth]{Figs/Idealista_dynamics/distance_attach.pdf}
	\caption[Attachment dynamics of new listings by distance.]{ Attachment dynamics of new listings by distance. \textbf{(a)} Agency radius (mean distance of the listings to the agency center of mass) as a function of the distance of the new listing (to the center of mass) attached to the agency. Different colors indicate different regions or provinces. The black dotted grey line shows a linear increase $R = d_{\rm new}$, while the dashed black line shows a square root dependence $R = C \, d_{\rm new}^{1/2}$ with $C = 120$. \textbf{(b)} Distribution of the logarithmic difference between the distance of the new listing and radius of the agency. Different colors indicate different regions or provinces. Each solid colored line corresponds to a T-student fit for each region. The parameters are: $\mu = 0.02$, $\sigma = 0.5$ for Madrid, $\mu = 0.02$, $\sigma = 0.5$ for Barcelona, and $\mu = 0.02$, $\sigma = 0.5$ for the Balearic Islands. \label{fig:distance_attach}}
\end{figure}

Regarding the dynamics, the location of a new listing also affects the attachment process to an agency. Fig. \ref{fig:distance_attach}(a) shows the mean agency radius as a function of the distance of the new listing attached to the agency. In this case, the scenario is very different to the price attachment. Here, the mean agency radius follows a proportional dependence to the distance listing - CM of the agency, but this dependence is clearly sublinear (similar to $R \sim d_{\rm new}^{1/2})$. Thus, the majority of listings are attached to agencies with a radius of action larger or equal than the distance between the listing and the agency CM. This result is consistent in all regions and is a relevant observation of the spatial specialization of agencies, inherent in the housing market dynamics. New listings are attached to agencies that operate in that specific area, so the location of a house is relevant factor in the decision-making process of the house owners.

Moreover, as we did for the pricing, we can compute the distribution of the logarithmic difference between the distance of the new listing and the mean distance of the agency. Now, this distribution is not centered at $0$, but at negative values bellow $-1$, as shown in Fig. \ref{fig:distance_attach}(b). This result suggests that the new listings are attached to agencies where the radius is equal or larger than the distance listing - CM but with present exceptions to this process. Notice that this distribution is not static, and the agency CM and radius are changing in time and so the attachment process,. 


\section{Conclusions}

In this work, we have analyzed the dynamics of the housing market in Spain, using a comprehensive dataset of listings posted on the online platform \texttt{Idealista.com}. We have shown that the listings' dynamics are non-stationary, with a weekly pattern in the number of active listings and a linear increase in the total number of listings. The listings' life distribution shows a fat tail, indicating a bursty dynamics, but the inter-activations time distribution is not as heterogeneous as in other human activities. We have also shown that agencies follow a preferential attachment mechanism, where the probability of a new listing being attached to an agency is proportional to the number of listings posted by the agency previously. This mechanism is correlated with the price of the listings, as the variance of the listings price is proportional to the mean price of the agency. The price of the listings is also correlated with the degree of the agency, showing a non-trivial dependence. The spatial correlations of the listings show that the variance of the distance between the agency center of mass and the listings is proportional to the mean distance of the agency. The distance of a new listing attached to an agency is also correlated with the mean distance of the agency, showing a sublinear dependence. This spatial correlation is relevant in the decision-making process of the house owners, as the location of a house is a key factor in the housing market dynamics.

Further work could explore the role of the agencies in the housing market dynamics, and how the decision-making process of the house owners is influenced by the agencies. The spatial correlations of the listings could be analyzed in more detail, exploring the spatial patterns of the listings and how they are correlated with the agencies. The temporal dynamics of the listings could also be explored, analyzing the weekly patterns in more detail and how they are correlated with the agencies. Finally, the role of the agencies in the housing market dynamics could be analyzed in more detail, exploring how the decision-making process of the house owners is influenced by the agencies and how the agencies are correlated with the listings.

Moreover, the results presented in this work could be used to develop models of the housing market dynamics, exploring how the listings are posted on the online platform and how the agencies influence the decision-making process of the house owners. These models could be used to predict the future evolution of the housing market and to explore how the agencies could influence the decision-making process of the house owners. The results presented in this work could also be used to develop models of the housing market dynamics, exploring how the listings are posted on the online platform and how the agencies influence the decision-making process of the house owners. These models could be used to predict the future evolution of the housing market and to explore how the agencies could influence the decision-making process of the house owners.











