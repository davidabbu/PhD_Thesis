This thesis has delved into the intricate dynamics of aging and memory effects in social and economic systems, particularly focusing on their implications in various threshold models and housing market dynamics. Here, we provide a comprehensive discussion of our findings and propose future research directions, addressing potential advancements in the study of aging and its broader applications.

In the context of aging, our research has shown significant impacts on the dynamics of different threshold models. In agreement with the aging implications in the Voter model~\cite{artime-2018,artime2019herding,peralta-2020C,peralta-2018,peralta-2020A}, we found that aging promotes ordering in phases dynamically disordered in threshold models. In both, the Sakoda-Schelling and the Symmetrical Threshold model, aging is able to induce coarsening in a phase that would otherwise remain disordered. This effect occurs because aging introduces a persistence of small clusters, which can grow and eventually dominate the system, leading to segregation or consensus, depending on the model. On the other hand, the coarsening process is found to be slower in the presence of aging. This result is observed in the 3 models analyzed: a lower exponent at the interface decay of the Sakoda-Schelling model, an exponential cascade replaced by a power-law in the Granovetter-Watts model and an exponential interface decay replaced by a power-law decay in the Symmetrical Threshold model. Following these results, aging can be seen as a complex mechanism that prevents small clusters to dissolve, in a halfway between order and disorder, allowing for them to grow in a disordered regime and dominate the system (promote order), but at the same time preventing ordering dynamics to reach the segregated/fully adopted/consensus state (slowing down the coarsening process). 

Moreover, aging mechanism was found to change dramatically the dynamics in threshold models and its effects are found to be robust across different update rules. Nevertheless, typical agent-based Monte Carlo simulations are still using a Random Asynchronous Update, which misses the non-Markovian nature of human interactions. Through this thesis, we have shown how aging is an essential ingredient for the modeling of social and economic systems, and thus, it is crucial to incorporate it into future models. 

The next steps for studying aging involve extending this mechanism to different models of collective behavior, as the ones studied through this thesis. One promising direction is the inclusion of aging effects in the idiosyncratic behavior. The idiosyncratic behavior is a key feature of human decision-making, where individuals exhibit a certain degree of randomness or irrationality in their choices. By incorporating aging into these models, the independent behavior also decays with persistence time, motivated as an irrationality decay with the memory of their past decisions. Additionally, in the context of complex and group interactions, future modifications could explore the joint effect of aging and higher-order interactions in complex systems, considering the interplay between individual memory and collective behavior. This approach can help uncover the deeper, often nonlinear, relationships that govern aging processes in social and economic systems. 

Another exciting direction for future research is the application of aging models to new domains. For instance, aging effects in financial markets could be investigated to understand how traders' prolonged engagement with specific assets influences market volatility and price trends. In the context of game theory, aging could be used to model how players' memory of past strategies affects their decision-making and the evolution of strategies over time~\cite{samuele-ciardella-2023}. In fact, it is found that aging can promote the evolution of cooperation in the Prisoner's Dilemma game~\cite{attila-2009}, and this mechanism could be extended to other social dilemmas or multi-agent systems to explore the role of memory in shaping collective behavior. The applications in this last domain could adress questions like how memory to past strategies could help to avoid the tragedy of the commons~\cite{ostrom1990governing}, or if it could help to avoid the escalation of conflicts~\cite{axelrod1981evolution}.

Moreover, the validation of our results with empirical data could provide further insights into the real-world implications of aging in social and economic systems. By analyzing historical data on segregation patterns, consumer behavior, or adoption curves, we can test the predictions of our models and refine our understanding of aging effects in complex systems. However, the challenge lies in obtaining high-quality data that captures the temporal evolution of these systems accurately and the personal engagement or persistence of the individuals involved, as a proxy for aging. Collaborations with industry partners or government agencies could help in accessing such data and conducting empirical studies to validate our theoretical findings. Another interesting direction to validate the aging mechanism is to conduct experiments with human subjects, where the persistence of the individuals is measured and correlated with their decisions. This could provide valuable insights into how aging influences different tasks or decision-making processes.

Regarding the housing market dynamics, our research suggests several future directions. Firstly, the current findings are based on our dataset, and it is crucial to validate these results with larger and more diverse datasets. This would help to generalize our conclusions and ensure their robustness across different contexts. An interesting future work could involve developing an activity-driven model for the housing market, borrowing ideas from models with aging and for the 3 key factors that drive the decision-making of sellers (popularity, price and location). In this sense, Part 1 of the thesis, which focuses on aging and memory effects in abstract models, is closely related to Part 2, which applies these concepts to the real-world scenario of housing market dynamics. The listings in an online platform exhibit aging, as observed in the persistence of listings in the online platform. In this context, aging is not understood as a memory effect, but as a time decaying attractiveness of the listings. However, the same tools and mathematical concepts used in Part 1 can be applied to build an activity driven model to understand the dynamics of the housing market, which might be possible to describe via differential equations via the AME formalism described in Chapter 4.

Regarding segmentation, further investigation is needed to understand the dependence on data scale. For example, analyzing housing data for an entire country at a granular level, such as 1 $\times$ 1 km$^2$ cells, could reveal whether the observed communities persist or if larger communities emerge within which our identified communities reside. This scale-dependent analysis could help in identifying broader market trends and regional dependencies that are not apparent at smaller scales. Moreover, investigating temporal changes in market segmentation could provide insights into how economic cycles, policy changes, or significant events influence market structure over time.

Future work along these lines could involve more detailed analysis of other metadata and their impact on market segmentation. For example, investigating the role of demographic factors, economic indicators, or urban development plans on market dynamics could provide a richer understanding of the forces shaping the housing market. Integrating machine learning techniques with our segmentation methodology could also enhance the ability to predict and respond to market changes in real-time.

Finally, the analysis of real systems, like the housing market, has significant implications for policymaking and price analysis. By focusing on the dynamics of listings rather than the aggregated data, our thesis offers a more realistic perspective on how real estate agencies operate in the real world, where equilibrium is rarely achieved. Understanding these dynamics is crucial for developing effective policies and strategies that can adapt to the ever-changing nature of social and economic systems.

In conclusion, this thesis contributed to the understanding of aging and memory effects in social and economic dynamics. Future research should continue to refine these models, validate findings with broader datasets, and explore new applications in real-world systems. By addressing the temporal nature of these systems, our research opens up new avenues for innovation in modeling and understanding the complex interplay of factors that drive socio-economic behavior.

\fancyhead[LO]{\sffamily\normalsize\bfseries Chapter 10. General conclusions and outlook}
